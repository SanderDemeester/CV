\documentclass{tccv}
\begin{document}
\part{Sander Demeester}
\section{Werk ervaring}
\begin{eventlist}
\item{2013 Juli - Augustug}
{HPC, Universiteit Gent}
{Software developer}

Software developer bij High performance en computing infrastructure aan de univesiteit gent.
\href{https://github.com/hpcugent}{HPC Github}

\item{2010-2011}
{Stage Provincie West-Vlaanderen dienst ICT}
{Administratief en ICT medewerker}
Mijn laatstejaar in het middelbaar liep ik stage voor administratief medewerken. Deze stage is geevolueerd naar ICT werk en programeren naarmate bleek dat ik een meerwaarde kon zijn bij de ICT dienst. Het resultaat was een security model met een advies over het WiFi Security Model(802.11i)
\end{eventlist}
\section{Onderwijs}
\begin{yearlist}
\item[Middelbaar onderwijs diploma]{2003 - 2010}
	 {Buitengewoon onderwijs autisme spectrum stoornis}
	 {Spermalie, BRUGGE}
\item[Informatica]{2010 - 2013}{Universiteit Gent}{Behaalde resultaten}
	\end{yearlist} 
\begin{factlist}
\item{1ste bachelor}{Programmeren 1 en 2}
\item{1ste bachelor}{Communicatie technieken}
\item{1ste bachelor}{Recht van de intelectuele eigendommen}
\item{1ste bachelor}{Computergebruik (geavanceerd linux gebruik)}
\item{1ste bachelor}{Scriptingtalenn (geavanceerd linux gebruik)}
\item{1,2,3 bachelor}{Algoritmes en Datastructuren}
\item{2de bachelor}{Software ontwikkeling 1, 2}
\item{2de bachlor}{Computerarchitctuur}
\item{3de bachelor}{Besturingssystemen}
\item{3de bachelor}{Communicatie netwerken}
\item{3de bachelor}{Vakoverschrijvend project}
\item{3de bachelor}{Internettechnologie}
\item{1ste master}{Informatie beveiliging en cryptografie}
\item{1ste master}{Studie van programeertalen en programeermodellen}
\end{factlist}

\section{Opleiding gerelateerde projecten}
\begin{yearlist}
\item{2012}
	 {AD3zip()}
	 {Eigen mplementatie van datacompressie algoritmes\newline
	 * Burrows Wheeler\newline
	 * lz77\newline
	 * Huffman}	 
\item{2013}
	{Avank-bebras}
	{Web applicatie om online competitie uit te voeren
	Vak overschrijvend project}
\item{2013}
	 {Project informatie beveiliging}
	 {Ontwikkelen van online betaal systeem, HOTP met android}
\end{yearlist}

\section{Niet opleiding gerelateerde projecten}
\begin{yearlist}
\item{2010-2013}
	 {ZeusWPI}
	 {Actief lid van ZeusWPI werkgroep (http://zeus.ugent.be)}
	 
\item{2011-2013}
	 {Werkgroep Ethical hacking}
	 {Actief lid werkgroep Ethical Hacking van de vakgroep ELIS bij Universiteit Gent}
	 
\item{2012-2013}
	 {Auteur van verschillende opensource projecten}
	 {Zie github.com/SanderDemeester\newline
	 * Network analyse tools \newline
	 * Eigen implementatie van cryptografische algoritmes \newline
	 (met als doel een beter begrip van hun werking)}
	 
\item{2009 --}
	{Alix 2D2 en raspberry PI}
	{Het maken van een homerouter met het OpenBSD systeem}
	
\item{2007 - 2013}
	{Bijwonen IT-gerelateerde conferenties}
	{* FOSDEM \newline
	 * BruCON \newline
	 * Blackberry developer conference \newline
	 * VPW}
\item{2013}
	{Cryptografie studiedag} 
	{Studie omtrent SHA3, en permutation based cryptografie georganiseerd door de VUB}
\item{2009 -- }
	{Lezen academische publiciaties}
	{IEEE en ACM}
	
\end{yearlist}
\newpage
\section{Computerkennis}
\subsection{Programeertalen}
\begin{itemize}
\item Java
\subitem JDBC
\subitem jbox2d
\subitem 
\item C
\item C++
\item Python
\item Javascript
\subitem sigma.js
\subitem node.js
\subitem faye.js
\item .NET
\subitem ASP.NET
\item x86 assembler en ARM assembly.
\subsubsection*{Frameworks en Library}
\item libxml2
\item ncurses
\item Play framework
\item Android (basis)
\item Sage Mathematics
\end{itemize}
\subsection{Systeembeheer}
\begin{itemize}
\item Linux en OpenBSD
\subitem DNS/BIND (dnssec)
\subitem Geavanceerde kennis van TCP/IP en netwerk technieken.
\subitem Goede kennis van IPv6 + deploy technieken
\subitem Kennis VPN technologieen. (TLS/IPSEC)
\subitem Routeer en Dispatch algoritmes (RIPv2, BGP basis).
\subitem Uistekende kennis van de twee grote firewall based rule engines (IPtables,PF).
\subitem Ervaring met NetFlow packet analysis.
\subitem Ervaring met Apache.
\subitem Ervaring met OpenSSL.
\subitem Network Engineering
\subitem Network Architecture
\item Ervaring met Windows, Linux (Debian based, Gentoo, CentOS) en OpenBSD.
\subitem Scripting, bash, sed en awk.
\end{itemize}
\subsection*{Tools}
\begin{itemize}
\item Wireshark
\item Postgresql
\item MySQL
\item Jenkins
\item Eclipse
\item Emacs
\end{itemize}
\subsection{Varia}
\begin{itemize}
\item Blog (http://www.sanderdemeester.be/blog)
\item Semantic web en Linked Data.
\item Version control
\subitem git.
\subitem basiskennis SVN.
\subitem basiskennis Mercurial SCM.
\item basiskennis databank normalisatie
\end{itemize}
\section{Persoonlijk}
Ikzelf leid aan het autisme spectrum stoornis, waardoor mijn opleidingsloopbaan afwijkt van het normale traject. Ik heb meegewerkt aan een project van artevelde hogeschool om autisme meer bekend te maken in alle vlaamse scholen. Ik heb een grote passie voor informatienetwerken/systemen en getallentheorie.

\end{document}