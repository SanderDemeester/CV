\documentclass[margin, 10pt]{res} % Use the res.cls style, the font size can be changed to 11pt or 12pt here


\usepackage{fullpage}
\usepackage{multicol}
\usepackage{helvet} % Default font is the helvetica postscript font
%\usepackage{newcent} % To change the default font to the new century schoolbook postscript font uncomment this line and comment the one above

\setlength{\textwidth}{5.1in} % Text width of the document

\begin{document}

%----------------------------------------------------------------------------------------
%	NAME AND ADDRESS SECTION
%----------------------------------------------------------------------------------------

\moveleft.5\hoffset\centerline{\large\bf Sander Demeester} % Your name at the top
 
\moveleft\hoffset\vbox{\hrule width\resumewidth height 1pt}\smallskip % Horizontal line after name; adjust line thickness by changing the '1pt'
 
\moveleft.5\hoffset\centerline{Dubbelsingel 6} % Your address
\moveleft.5\hoffset\centerline{Brugge, Sint-kruis}
\moveleft.5\hoffset\centerline{04 76 86 48 84}
\moveleft.5\hoffset\centerline{demeester.sander@gmail.com}
\moveleft.5\hoffset\centerline{www.sanderdemeester.be}
\moveleft.5\hoffset\centerline{1990/01/17}

%----------------------------------------------------------------------------------------

\begin{resume}

%----------------------------------------------------------------------------------------
%	OBJECTIVE SECTION
%----------------------------------------------------------------------------------------
 
\section{Persoonlijk}  
Ikzelf leid aan het autisme spectrum stoornis, waardoor mijn opleidingsloopbaan afwijkt van het normale traject. Ik heb een grote passie voor informatienetwerken/systemen en getallentheorie.

%----------------------------------------------------------------------------------------
%	Werk ervaring
%---------------------------------------------------------------------------------------- 
\section{Werkervaring}
{\sl Software developer} \hfill Zomerstage 2013 \\
Ontwikkelen van Python VSC programma voor analyseren van grote hoeveelheid log informatie en statische data-analyse/machine learning. Implementeren en toepassen van algoritmes om waardevolle informatie te kunnen extraheren en verwerken.
Bij High performance en computing infrastructure aan de Universiteit van Gent.\\ (github.com/hpcugent)

{\sl Administratief en ICT medewerker} \hfill 2010 - 2011 \\
Stage Provincie West-Vlaanderen dienst ICT. \\
Mijn laatste jaar in het middelbaar liep ik stage voor administratief medewerker. Deze stage is ge\"evolueerd naar ICT werk en programmeren naarmate bleek dat ik een meerwaarde kon zijn bij de ICT dienst. Het resultaat was een security model met een advies over het WiFi Security Model(802.11i).

%----------------------------------------------------------------------------------------
%	EDUCATION SECTION
%----------------------------------------------------------------------------------------
\section{Opleiding}

{\sl Middelbaar onderwijs,} Buitengewoons onderwijs autisme spectrum stoornis \\
Spermalie BRUGGE (2002 - 2010)
\\ \\
{\sl Informatica,} Universiteit Gent \\
Behaalde resultaten:
\begin{tabular}{l l l}
1ste bachelor (2010-2011) & Programmeren 1 en 2 \\
						  & Communicatie technieken \\
						  & Recht van intellectuele eigendommen \\
						  & Computergebruik (geavanceerd linux gebruik) \\
						  & Scriptingtalen (geavanceerd linux gebruik) \\
\\					 
1,2,3 bachelor (2010-2013) &  Algoritmes en Datastructuren \\
\\
2de bachelor (2011-2012) & Software ontwikkeling 1 en 2 \\
		                 & Computerarchitctuur \\
\\	
3de bachelor (2012-2013) & Besturingssystemen \\
						 & Communicatienetwerken \\
						 & Vakoverschrijvend project \\
						 & Internettechnologie \\
1ste master (2012-2013)  & Informatie beveiliging en cryptografie \\
						 & Studie van programeertalen en modellen \\
\end{tabular}
\section{Talenkennis}
Nederlands (Moedertaal)
Engels (Lezen, schrijven en spreken zijn vloeiend)
\pagebreak
%----------------------------------------------------------------------------------------
% Opleiding gerelateerde projecten.
%----------------------------------------------------------------------------------------

\section{Opleiding gerelateerde \\ projecten} 

{\sl Algoritmes \& Datastructuren:} AD3zip, ZW-tree \\
Eigen implementatie van datacompressie algoritmes
\begin{itemize}
\item Burrows Wheeler
\item lz77
\item Huffman
\end{itemize}
{\sl Avank-bebras:} Vakoverschrijvend project \\
Systeem administrator \& Software developer. \\
http://github.com/SanderDemeester/bebras-avank.
\\ \\
{\sl Project informatiebeveiliging:} Ontwerpen en implementeren van online betaal systeem. Multi-factor authenticatie (HOTP) met android.
 
%----------------------------------------------------------------------------------------
%	Niet opleiding gerelateerde projecten
%----------------------------------------------------------------------------------------
 
\section{Niet opleiding\\ gerelateerde projecten}

{\sl Alix 2D2 en raspberry PI} \\
Het maken en beheren/configureren van homerouter met OpenBSD. \\
Bouwen van raspberry PI cluster.

{\sl Auteur van verschillende opensource projecten} \\
zie github.com/SanderDemeester
\begin{itemize} \itemsep -2pt
\item Network analysis tools
\item Eigen implementatie van cryptografische algoritmes (met als doel een beter begrip van hun werking)
\end{itemize}

{\sl ZeusWPI} \\
Actief lid van ZeusWPI, werkgroep informatie aan de Universiteit van Gent.
\begin{itemize} \itemsep -2pt
\item{http://zeus.ugent.be}
\item{http://github.com/ZeusWPI}
\end{itemize}

{\sl Werkgroep Ethical hacking} \\
Actief lid werkgroep Ethical Hacking van de vakgroep ELIS bij universiteit Gent.

{\sl Bijwonen van IT-gerelateerde conferenties}
\begin{itemize} \itemsep -2pt
\item FOSDEM
\item BruCON
\item VPW
\item Blackberry developer conference
\end{itemize}

{\sl Cryptografie studiedag} \\
Studie omtrent SHA3, en permutation based cryptografie georganiseerd door de VUB.

{\sl lezen van internationale academische publicaties van IEEE en ACM.}

%----------------------------------------------------------------------------------------
%	Kennis sectie
%---------------------------------------------------------------------------------------- 
\end{resume}
\pagebreak
\section{Computerkennis}
\subsection{Programeertalen}
\begin{multicols}{2}
\begin{enumerate}
\item[-] Java
\subitem JDBC
\subitem jbox2d
\subitem Swing
\item[] C / C++
\item[] Python
\item[] Javascript
\subitem sigma.js
\subitem node.js
\item[] .NET
\subitem ASP.NET
\item[] x86 assembler en ARM assembly.
\subsubsection*{Frameworks en Library}
\item[] libxml2
\item[] ncurses
\item[] Play framework
\item[] Android (basis)
\item[] Sage Mathematics
\item[] scikit-learn
\end{enumerate}
\end{multicols}
\subsection{Systeembeheer}
\begin{itemize}
\item Linux en OpenBSD
\subitem DNS/BIND (dnssec)
\subitem Geavanceerde kennis van TCP/IP en netwerktechnieken.
\subitem Goede kennis van IPv6 + deploy technieken
\subitem Kennis VPN technologieen. (TLS/IPSEC)
\subitem Routeer en Dispatch algoritmes (RIPv2, BGP basis).
\subitem Uistekende kennis van de twee grote firewall based rule engines (IPtables,PF).
\subitem Ervaring met NetFlow packet analysis.
\subitem Ervaring met Apache.
\subitem Ervaring met OpenSSL.
\subitem Network Engineering.
\subitem Network Architecture.
\item Ervaring met Windows, Linux (Debian based, Gentoo, CentOS en BackTrack) en OpenBSD.
\subitem Scripting, bash, sed en awk.
\end{itemize}
\begin{multicols}{2}
\subsection*{Tools}
\begin{enumerate}
\item[] Wireshark
\item[] Postgresql
\item[] MySQL
\item[] Graylog2
\item[] Logstash
\item[] Jenkins
\item[] ZAP
\item[] Eclipse
\end{enumerate}
\subsection{Varia}
\begin{enumerate}
\item[] Blog (http://www.sanderdemeester.be/blog)
\item[] Semantic web en Linked Data.
\item[] Continuous integration.
\item[] OWASP
\item[] Version control
\subitem git.
\subitem basiskennis SVN.
\subitem basiskennis Mercurial SCM.
\item[] basiskennis databank normalisatie
\end{enumerate}
\end{multicols}
\section{Referenties}
{\sl Luis Fernando Munoz \hfill HPC Systems engineer bij Universiteit Gent. \\
luis.munoz@ugent.be
}

{\sl Jens Timmerman \hfill System Engineer HPC bij Universiteit Gent. \\
jens.timmerman@ugent.be
}

{\sl Kenneth Hoste \hfill HPC system administrator bij G Universiteit Gent. \\
kenneth.hoste@ugent.be
}

{\sl Jeroen De Wachter \hfill IT Consultant bij Addestino  \\
jeroen.de.wachter@telenet.be
}
\end{document}