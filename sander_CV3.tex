\documentclass[margin, 10pt]{res} % Use the res.cls style, the font size can be changed to 11pt or 12pt here


\usepackage{fullpage}
\usepackage{multicol}
\usepackage{helvet} % Default font is the helvetica postscript font
%\usepackage{newcent} % To change the default font to the new century schoolbook postscript font uncomment this line and comment the one above

\setlength{\textwidth}{5.1in} % Text width of the document

\begin{document}

%----------------------------------------------------------------------------------------
%	NAME AND ADDRESS SECTION
%----------------------------------------------------------------------------------------

\moveleft.5\hoffset\centerline{\large\bf Sander Demeester} % Your name at the top
 
\moveleft\hoffset\vbox{\hrule width\resumewidth height 1pt}\smallskip % Horizontal line after name; adjust line thickness by changing the '1pt'
 
\moveleft.5\hoffset\centerline{Dubbelsingel 6} % Your address
\moveleft.5\hoffset\centerline{Brugge, Sint-kruis}
\moveleft.5\hoffset\centerline{(04)76 86 48 84}

%----------------------------------------------------------------------------------------

\begin{resume}

%----------------------------------------------------------------------------------------
%	OBJECTIVE SECTION
%----------------------------------------------------------------------------------------
 
\section{Persoonlijk}  
Ikzelf leid aan het autisme spectrum stoornis, waardoor mijn opleidingsloopbaan afwijkt van het normale traject. Ik heb meegewerkt aan een project van artevelde hogeschool om autisme meer bekend te maken in alle vlaamse scholen. Ik heb een grote passie voor informatienetwerken/systemen en getallentheorie.

%----------------------------------------------------------------------------------------
%	EDUCATION SECTION
%----------------------------------------------------------------------------------------

\section{Opleiding}

{\sl Middelbaar onderwijs,} Buitengewoons onderwijs autisme spectrum stoornis \\
Spermalie BRUGGE (2002 - 2010)
\\ \\
{\sl Informatica,} Universiteit Gent \\
Behaalde resultaten:

\begin{itemize}
\item{1ste bachelor : Programmeren 1 en 2}
\item{1ste bachelor : Communicatie technieken}
\item{1ste bachelor : Recht van de intelectuele eigendommen}
\item{1ste bachelor : Computergebruik (geavanceerd linux gebruik)}
\item{1ste bachelor : Scriptingtalenn (geavanceerd linux gebruik)}
\item{1,2,3 bachelor : Algoritmes en Datastructuren}
\item{2de bachelor : Software ontwikkeling 1, 2}
\item{2de bachlor : Computerarchitctuur}
\item{3de bachelor : Besturingssystemen}
\item{3de bachelor : Communicatie netwerken}
\item{3de bachelor : Vakoverschrijvend project}
\item{3de bachelor : Internettechnologie}
\item{1ste master : Informatie beveiliging en cryptografie}
\item{1ste master: Studie van programeertalen en programeermodellen}
\end{itemize}
 
%----------------------------------------------------------------------------------------
% Opleiding gerelateerde projecten.
%----------------------------------------------------------------------------------------

\section{Opleiding gerelateerde \\ Projecten} 

{\sl Algoritmes \& Datastructuren:} AD3zip, ZW-tree \\
Eigen implementatie van datacompressie algoritmes
\begin{itemize}
\item Burrows Wheeler
\item lz77
\item Huffman
\end{itemize}
{\sl Avank-bebras:} Vakoverschrijvend project \\
Systeem administrator \& Software developer. \\
http://github.com/SanderDemeester/bebras-avank.
\\ \\
{\sl Project informatiebeveiliging:} Ontwerpen en implementeren van online betaal systeem. Multie-factor authenticatie (HOTP) met android.
 
%----------------------------------------------------------------------------------------
%	Niet opleiding gerelateerde projecten
%----------------------------------------------------------------------------------------
 
\section{Niet opleiding\\ gerelateerde projecten}

{\sl Alix 2D2 en raspberry PI} \hfill 2009 - huidig  \\
Het maken en beheren/configureren van homerouter met het OpenBSD systeem.

{\sl Auteur van verschillende opensource projecten} \hfill 2011 - 2013  \\
zie github.com/SanderDemeester
\begin{itemize} \itemsep -2pt
\item Network analysis tools
\item Eigen implementatie van cryptografische algoritmes (met als doel een beter begrip van hun werking)
\end{itemize}

{\sl ZeusWPI} \hfill 2010 - 2013 \\
Actief lid van ZeusWPI, werkgroep informatie aan de Universiteit van Gent.
\begin{itemize} \itemsep -2pt
\item{http://zeus.ugent.be}
\item{http://github.com/ZeusWPI}
\end{itemize}

{\sl Werkgroep Ethical hacking} \hfill 2011 - 2013 \\
Actief lid werkgroep Ethical Hacking van de vakgroep ELIS bij universiteit Gent.

{\sl Bijwonen van IT-gerelateerde confefrenties} \hfill 2007 - 2013 \\
\begin{itemize} \itemsep -2pt
\item FOSDEM
\item BruCON
\item VPW
\item Blackberry developer conference
\end{itemize}

{\sl Cryptografie studiedag} \hfill 2013 \\
Studie omtrent SHA3, en permutation based cryptografie georganiseerd door de VUB.

{\sl Lezen academische publiciaties} \hfill 2009 - huidig \\
IEEE en ACM.

{\sl Assistant Manager} \hfill Summers 1988-89 \\
Thunder Restaurant, Canton, CT
\begin{itemize}
\item Recognized need for, developed, and wrote employee training manual. Performed various duties including cooking, employee training, ordering, and inventory control. 
\end{itemize} 
%----------------------------------------------------------------------------------------
%	Werk ervaring
%---------------------------------------------------------------------------------------- 
\section{Werk ervaring}
{\sl Software developer} \hfill Zomerstage 2013 \\
Ontwikkelen van pyrthon module voor analyseren van logs en grafische verwerking in python\\
Bij High performance en computing infrastructure aan de univesiteit gent. (github.com/hpcugent)

{\sl Administrafief en ICT medewerker} \hfill 2010 - 2011 \\
Stage Provincie West-Vlaanderen dienst ICT. \\
Mijn laatstejaar in het middelbaar liep ik stage voor administratief medewerken. Deze stage is geevolueerd naar ICT werk en programeren naarmate bleek dat ik een meerwaarde kon zijn bij de ICT dienst. Het resultaat was een security model met een advies over het WiFi Security Model(802.11i).
%----------------------------------------------------------------------------------------
%	Kennis sectie
%---------------------------------------------------------------------------------------- 
\end{resume}
\section{Computerkennis}
\subsection{Programeertalen}
\begin{multicols}{2}
\begin{enumerate}
\item[-] Java
\subitem JDBC
\subitem jbox2d
\subitem Swing
\item[] C / C++
\item[] Python
\item[] Javascript
\subitem sigma.js
\subitem node.js
\item[] .NET
\subitem ASP.NET
\item[] x86 assembler en ARM assembly.
\subsubsection*{Frameworks en Library}
\item[] libxml2
\item[] ncurses
\item[] Play framework
\item[] Android (basis)
\item[] Sage Mathematics
\end{enumerate}
\end{multicols}
\subsection{Systeembeheer}
\begin{itemize}
\item Linux en OpenBSD
\subitem DNS/BIND (dnssec)
\subitem Geavanceerde kennis van TCP/IP en netwerk technieken.
\subitem Goede kennis van IPv6 + deploy technieken
\subitem Kennis VPN technologieen. (TLS/IPSEC)
\subitem Routeer en Dispatch algoritmes (RIPv2, BGP basis).
\subitem Uistekende kennis van de twee grote firewall based rule engines (IPtables,PF).
\subitem Ervaring met NetFlow packet analysis.
\subitem Ervaring met Apache.
\subitem Ervaring met OpenSSL.
\subitem Network Engineering
\subitem Network Architecture
\item Ervaring met Windows, Linux (Debian based, Gentoo, CentOS) en OpenBSD.
\subitem Scripting, bash, sed en awk.
\end{itemize}
\begin{multicols}{2}
\subsection*{Tools}
\begin{enumerate}
\item[] Wireshark
\item[] Postgresql
\item[] MySQL
\item[] Jenkins
\item[] Eclipse
\item[] Emacs
\end{enumerate}
\subsection{Varia}
\begin{enumerate}
\item[] Blog (http://www.sanderdemeester.be/blog)
\item[] Semantic web en Linked Data.
\item[] Version control
\subitem git.
\subitem basiskennis SVN.
\subitem basiskennis Mercurial SCM.
\item[] basiskennis databank normalisatie
\end{enumerate}
\end{multicols}{2}
\end{document}